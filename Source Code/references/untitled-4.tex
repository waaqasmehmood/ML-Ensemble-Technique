\documentclass[journal, twoside, final]{IEEEtran}
\usepackage{geometry}
\usepackage{graphicx}
\usepackage{multirow}

% Adjust the page layout
\geometry{margin=1in}

\begin{document}

\begin{figure*}[!ht]
  \centering
  \caption{Attribute Description}
  \begin{tabular}{|c|l|p{5cm}|l|}
    \hline
    \textbf{No} & \textbf{Attribute} & \textbf{Description} & \textbf{Values} \\
    \hline
    1 & age & Age of patient & Age numeric values \\
    2 & gender & Gender & 1-Male, 0-Female \\
    3 & C\_P & Chest-pain & 1 = typ-ang, 2 = atypAng, 3 = non-ang \\
    4 & trest\_bps & Blood-pressure in rest & BP values \\
    5 & CHOL & Dietary fat & Cholesterol values \\
    6 & F-B-S & Sugar in blood & 1 = present, 0 = Absent \\
    7 & restecg & Echocardiography at rest & 1=Abnormal-ECG, 0=Normal \\
    8 & oldpeak & Exercise related to rest & Different values \\
    9 & slope & ST depression slope & 0, 1, 2 represent different slopes \\
    10 & CA & Vessels & 0, 1, 2, 3 values represent how your arteries are affected \\
    11 & thal & Thalassemia & 0, 1, 2, 3 represent colored Fluoroscopy vessels \\
    12 & thalach & Patient maximum heart rate & Heart rate values \\
    13 & exang & Angina with exercise & 1=present, 0=absent \\
    14 & target & Heart Disease & 1=effected person, 0=healthy \\
    \hline
  \end{tabular}
\end{figure*}

\end{document}
